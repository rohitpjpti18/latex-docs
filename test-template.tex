\documentclass[10pt, a4paper, twoside]{exam}
\usepackage{graphicx}
\usepackage[T1]{fontenc}
\usepackage{lmodern}
\usepackage[left=0.5in, top=0.5in, total={7.27in, 10.3in}]{geometry}
\usepackage[]{ulem}
\usepackage{blindtext}
\usepackage{multicol}
\setlength{\columnsep}{0.8cm}
\setlength{\columnseprule}{0.75pt}
\usepackage[utf8]{inputenc}
\usepackage[english]{babel}
\usepackage{fancyhdr}
\pagestyle{fancy}
\fancyhf{}
\fancyfoot[LE,RO]{ Green Valley School | Page \thepage}

\renewcommand{\headrulewidth}{0pt}
\renewcommand{\footrulewidth}{0.75pt}


\renewcommand\ULthickness{0.75pt}   %%---> For changing thickness of underline
\setlength\ULdepth{1ex} %\maxdimen ---> For changing depth of underline

\begin{document}
\fontfamily{lmr}\selectfont
\noindent \huge \textbf{Periodic Class Test - 1}

\vspace{0.03in}
\noindent \normalsize Academic Session 2022-23

\vspace{0.03in}
\noindent Class XI - Physics - Units and measurement

\vspace{0.03in}
\noindent Paper Code - SOMECODE123




\par
\vspace{0.1in}
\noindent
\textbf{\normalsize Maximum Time: 2 hours}\hfill \normalsize \textbf{Tuesday, 01 March 2022} 	\hfill \normalsize \textbf{Maximum mark: 35}
\vspace{0.05in}
\hline{\noalign \hfill}

\noindent

\begin{multicols*}{2}
    
    \noindent \textbf{\textit{General Instructions:}}
    \begin{itemize}
      \setlength{\itemsep}{1pt}
      \setlength{\parskip}{0pt}
      \setlength{\parsep}{0pt}
      \item \textit{There are 12 questions in all. All questions are compulsory.}
      \item \textit{This question paper has three sections: Section A, Section B and Section C.}
      \item \textit{Section A contains three questions of two marks each, Section B contains eight questions of three marks each, and Section C contains one case study-based question of five marks.}
      \item \textit{There is no overall choice. However, an internal choice has been provided in one question of two marks and two questions of three marks. You have to attempt only one of the choices in such questions.}
      \item \textit{You may use log tables if necessary but use of calculator is not allowed.}
    \end{itemize}
    \hline{\hfill}
    \noindent 
    
    \begin{questions}
    
        \question Draw Stress-strain graph for a metallic solid and show
    
        \begin{parts}
            \setlength{\itemsep}{1pt}
            \setlength{\parskip}{0pt}
            \setlength{\parsep}{0pt}
            \part elastic limit, 
            \part region of plasticity
            \part point of ultimate tensile strength and
            \part Fracture point on the graph.
        \end{parts}
    
        \question 100g of ice at 0°C is mixed with 100g of water at 80°C. The resulting temperature of the mixture is 6°C. Calculate the latent heat of fusion of ice.
        
        \question Draw Stress-strain graph for a metallic solid and show
    
        \begin{parts}
            \setlength{\itemsep}{1pt}
            \setlength{\parskip}{0pt}
            \setlength{\parsep}{0pt}
            \part elastic limit, 
            \part region of plasticity
            \part point of ultimate tensile strength and
            \part Fracture point on the graph.
        \end{parts}
    
        \question 100g of ice at 0°C is mixed with 100g of water at 80°C. The resulting temperature of the mixture is 6°C. Calculate the latent heat of fusion of ice.
        
        \question Draw Stress-strain graph for a metallic solid and show
    
        \begin{parts}
            \setlength{\itemsep}{1pt}
            \setlength{\parskip}{0pt}
            \setlength{\parsep}{0pt}
            \part elastic limit, 
            \part region of plasticity
            \part point of ultimate tensile strength and
            \part Fracture point on the graph.
        \end{parts}
    
        \question 100g of ice at 0°C is mixed with 100g of water at 80°C. The resulting temperature of the mixture is 6°C. Calculate the latent heat of fusion of ice.
        
        \question Draw Stress-strain graph for a metallic solid and show
    
        \begin{parts}
            \setlength{\itemsep}{1pt}
            \setlength{\parskip}{0pt}
            \setlength{\parsep}{0pt}
            \part elastic limit, 
            \part region of plasticity
            \part point of ultimate tensile strength and
            \part Fracture point on the graph.
        \end{parts}
    
        \question 100g of ice at 0°C is mixed with 100g of water at 80°C. The resulting temperature of the mixture is 6°C. Calculate the latent heat of fusion of ice.
        
        \question Draw Stress-strain graph for a metallic solid and show
    
        \begin{parts}
            \setlength{\itemsep}{1pt}
            \setlength{\parskip}{0pt}
            \setlength{\parsep}{0pt}
            \part elastic limit, 
            \part region of plasticity
            \part point of ultimate tensile strength and
            \part Fracture point on the graph.
        \end{parts}
    
        \question 100g of ice at 0°C is mixed with 100g of water at 80°C. The resulting temperature of the mixture is 6°C. Calculate the latent heat of fusion of ice.
        
        \question Draw Stress-strain graph for a metallic solid and show
    
        \begin{parts}
            \setlength{\itemsep}{1pt}
            \setlength{\parskip}{0pt}
            \setlength{\parsep}{0pt}
            \part elastic limit, 
            \part region of plasticity
            \part point of ultimate tensile strength and
            \part Fracture point on the graph.
        \end{parts}
    
        \question 100g of ice at 0°C is mixed with 100g of water at 80°C. The resulting temperature of the mixture is 6°C. Calculate the latent heat of fusion of ice.
        
        \question Draw Stress-strain graph for a metallic solid and show
    
        \begin{parts}
            \setlength{\itemsep}{1pt}
            \setlength{\parskip}{0pt}
            \setlength{\parsep}{0pt}
            \part elastic limit, 
            \part region of plasticity
            \part point of ultimate tensile strength and
            \part Fracture point on the graph.
        \end{parts}
    
        \question 100g of ice at 0°C is mixed with 100g of water at 80°C. The resulting temperature of the mixture is 6°C. Calculate the latent heat of fusion of ice.
        
        \question Draw Stress-strain graph for a metallic solid and show
    
        \begin{parts}
            \setlength{\itemsep}{1pt}
            \setlength{\parskip}{0pt}
            \setlength{\parsep}{0pt}
            \part elastic limit, 
            \part region of plasticity
            \part point of ultimate tensile strength and
            \part Fracture point on the graph.
        \end{parts}
    
        \question 100g of ice at 0°C is mixed with 100g of water at 80°C. The resulting temperature of the mixture is 6°C. Calculate the latent heat of fusion of ice.
        
        \question Draw Stress-strain graph for a metallic solid and show
    
        \begin{parts}
            \setlength{\itemsep}{1pt}
            \setlength{\parskip}{0pt}
            \setlength{\parsep}{0pt}
            \part elastic limit, 
            \part region of plasticity
            \part point of ultimate tensile strength and
            \part Fracture point on the graph.
        \end{parts}
    
        \question 100g of ice at 0°C is mixed with 100g of water at 80°C. The resulting temperature of the mixture is 6°C. Calculate the latent heat of fusion of ice.
        
        \question Draw Stress-strain graph for a metallic solid and show
    
        \begin{parts}
            \setlength{\itemsep}{1pt}
            \setlength{\parskip}{0pt}
            \setlength{\parsep}{0pt}
            \part elastic limit, 
            \part region of plasticity
            \part point of ultimate tensile strength and
            \part Fracture point on the graph.
        \end{parts}
    
        \question 100g of ice at 0°C is mixed with 100g of water at 80°C. The resulting temperature of the mixture is 6°C. Calculate the latent heat of fusion of ice.
        
        \question Draw Stress-strain graph for a metallic solid and show
    
        \begin{parts}
            \setlength{\itemsep}{1pt}
            \setlength{\parskip}{0pt}
            \setlength{\parsep}{0pt}
            \part elastic limit, 
            \part region of plasticity
            \part point of ultimate tensile strength and
            \part Fracture point on the graph.
        \end{parts}
    
        \question 100g of ice at 0°C is mixed with 100g of water at 80°C. The resulting temperature of the mixture is 6°C. Calculate the latent heat of fusion of ice.
        
        \question Draw Stress-strain graph for a metallic solid and show
    
        \begin{parts}
            \setlength{\itemsep}{1pt}
            \setlength{\parskip}{0pt}
            \setlength{\parsep}{0pt}
            \part elastic limit, 
            \part region of plasticity
            \part point of ultimate tensile strength and
            \part Fracture point on the graph.
        \end{parts}
    
        \question 100g of ice at 0°C is mixed with 100g of water at 80°C. The resulting temperature of the mixture is 6°C. Calculate the latent heat of fusion of ice.
        
        \question Draw Stress-strain graph for a metallic solid and show
    
        \begin{parts}
            \setlength{\itemsep}{1pt}
            \setlength{\parskip}{0pt}
            \setlength{\parsep}{0pt}
            \part elastic limit, 
            \part region of plasticity
            \part point of ultimate tensile strength and
            \part Fracture point on the graph.
        \end{parts}
    
        \question 100g of ice at 0°C is mixed with 100g of water at 80°C. The resulting temperature of the mixture is 6°C. Calculate the latent heat of fusion of ice.
        
        \question Draw Stress-strain graph for a metallic solid and show
    
        \begin{parts}
            \setlength{\itemsep}{1pt}
            \setlength{\parskip}{0pt}
            \setlength{\parsep}{0pt}
            \part elastic limit, 
            \part region of plasticity
            \part point of ultimate tensile strength and
            \part Fracture point on the graph.
        \end{parts}
    
        \question 100g of ice at 0°C is mixed with 100g of water at 80°C. The resulting temperature of the mixture is 6°C. Calculate the latent heat of fusion of ice.
        
    \end{questions}

\end{multicols*}
\end{document}