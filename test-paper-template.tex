\documentclass[10pt, a4paper]{exam}
\usepackage[left=0.5in, top=0.5in, total={7.27in, 10.1in}]{geometry}
\usepackage{ulem}
\usepackage{graphicx}
\usepackage{multicol}
\setlength{\columnsep}{0.8cm}
\setlength{\columnseprule}{0.50pt}
   %%---> For changing thickness of underline
\setlength\ULdepth{1.5ex}

\chead{}
\lfoot{}
\cfoot{}
\rfoot{\small \hrule \vspace{0.15cm} \hfill \textbf{Green Valley School $\mid$ Balaghat $\mid$} www.gvs.ac.in $\mid$ page \thepage}

\begin{document}

	\noindent 
    \begin{minipage}[l]{0.7\textwidth}
		\noindent \Huge \textbf{Periodic Class Test} \\
		\noindent \normalsize Academic Session 2022-23 \\
		\noindent Class XI $|$ Physics $|$ Units and measurement \\
		\noindent Paper Code - SOMECODE123 
	\end{minipage}
	\hfill\hfill
	\begin{minipage}[r]{0.3\textwidth}
			\normalsize	Roll no: \begin{tabular}{ |c|c|c|c|c|c|c|c| } 
 			\hline
 			$\;$ &$\;$ &$\;$ &$\;$ &$\;$ &$\;$ &$\;$ &$\;$ \\[5pt]
 			\hline
			\end{tabular} \vspace{0.05in}\\
			\normalsize	Name:\enspace\hrulefill
	\end{minipage}
	 
	
	\vspace{0.1in}
	\noindent \uline{\textbf{\normalsize Maximum Time: 2 hours}\hfill \normalsize 	\textbf{Tuesday, 01 March 2022} 	\hfill \normalsize \textbf{Maximum mark: 35}}
	
	\begin{multicols*}{2}
    
    \noindent \textbf{\textit{General Instructions:}}
    \begin{itemize}
      \setlength{\itemsep}{1pt}
      \setlength{\parskip}{0pt}
      \setlength{\parsep}{0pt}
      \item \textit{There are 12 questions in all. All questions are compulsory.}
      \item \textit{This question paper has three sections: Section A, Section B and Section C.}
      \item \textit{Section A contains three questions of two marks each, Section B contains eight questions of three marks each, and Section C contains one case study-based question of five marks.}
      \item \textit{There is no overall choice. However, an internal choice has been provided in one question of two marks and two questions of three marks. You have to attempt only one of the choices in such questions.}
      \item \textit{You may use log tables if necessary but use of calculator is not allowed.}
    \end{itemize}
    \hrule{\hfill}
    \noindent 
    
    \begin{questions}
    
        \question Draw Stress-strain graph for a metallic solid and show
    
        \begin{parts}
            \setlength{\itemsep}{1pt}
            \setlength{\parskip}{0pt}
            \setlength{\parsep}{0pt}
            \part elastic limit, 
            \part region of plasticity
            \part point of ultimate tensile strength and
            \part Fracture point on the graph.
        \end{parts}
        
        \question Write the number of significant digits in the following
    
        \begin{parts}
            \setlength{\itemsep}{1pt}
            \setlength{\parskip}{0pt}
            \setlength{\parsep}{0pt}
            \part 1001
            \part 100.1
            \part 0.001001
            \part 100.10
        \end{parts}
        
        \question State the number of significant figures in the following
    
        \begin{parts}
            \setlength{\itemsep}{1pt}
            \setlength{\parskip}{0pt}
            \setlength{\parsep}{0pt}
            \part $0.007 \; m^2$
            \part $2.64 \times 10^{24} \; m^2$ 
            \part $ 0.2370 \; g/cm^{-2}$
        \end{parts}
        
        \question Round the following numbers to 2 significant digits
    
        \begin{parts}
            \setlength{\itemsep}{1pt}
            \setlength{\parskip}{0pt}
            \setlength{\parsep}{0pt}
            \part $3472$
            \part $2.55$ 
            \part $84.16$
            \part $28.5$
        \end{parts}
        
        \question Write down the number of significant figures in the following:
    
        \begin{parts}
            \setlength{\itemsep}{1pt}
            \setlength{\parskip}{0pt}
            \setlength{\parsep}{0pt}
            \part $6428$
            \part $62.00 \; m$ 
            \part $0.00628 \; cm$
            \part $1200 \; N$
        \end{parts}
        
        \question The mass of a box measured by a grocer's balance is $2.300 \; kg$. Two gold pieces of masses $20.15 \; g$ and $20.17 \; g$ are added to the box. What is 
        
        \begin{parts}
            \setlength{\itemsep}{1pt}
            \setlength{\parskip}{0pt}
            \setlength{\parsep}{0pt}
            \part the total mass of the box,
            \part the difference in the masses of the pieces to correct significant figures?
        \end{parts}
        
        \question Length, breadth and thickness of a rectangular slab are $4.234 \; m, 1.005 \; m $ and $ 2.01 \; m$ respectively. Find volume of the slab to correct significant figures.
        
        \question A physical quantity $\rho$ is related to four variables $\alpha, \; \beta, \; \gamma$ and $\eta$ as: \[ \rho = \frac{\alpha^3\beta^2}{\eta\sqrt{\gamma}} \] The percentage errors of measurements in $\alpha, \; \beta, \; \gamma$ and $\eta$ are $1\%,\; 3\%, \;4\%$ and $2\%$ respectively. Find the percentage error in $\rho$.
        
        \question The period of oscillation of a simple pendulum is $T=2\pi\sqrt{L⁄g}$. Measured value of $L$ is $20.0\; cm$ known to $1\; mm$ accuracy and time for 100 oscillations of the pendulum is found to be 90$s$ using a wrist watch of $1s$ resolution. What is the accuracy in the determination of $g$?
        
		\question Find the percentage error in specific resistance given by \[\rho = \frac{\pi r^{2}R}{l}\] where $r$ is the radius having value $(0.2 \pm 0.02) \;cm$, $R$ is the resistance of $(60 \pm 2) \;ohm$ and $l$ is the length of $(150 \pm 0.1)\; cm$.      
		
		\question The resistance \[ R=\frac{V}{I}\] where $V = (100 \pm 5.0) \;V$ and $I = (10 \pm 0.2) \;A$. Find the percentage error in $R$.
		
		\question A thin wire has length of $21.7 \;cm$ and radius $0.46 \;mm$. Calculate the volume of the wire to correct significant figures.
 		
 		\question A cube has a side of length $2.342 \;m$. Find volume and surface area in correct significant figures.
 		
 		\question Add $6.75 \times 10 \;cm^{3}$ to $4.52 \times 10 \;cm^{2}$ with regard to significant figures.
        
        \question The temperature of two bodies measured by a thermometer are $(20 \pm 0.5)^{\circ}C$ and $(50 \pm 0.5)^{\circ}C$. Calculate the temperature difference with error limits.
        
        \question The radius of a sphere is measured to be $(2.1 \pm 0.5) \;cm$. Calculate its surface area with error limits.
        
        \question Find density when a mass of $9.23 \;kg$ occupies a volume of $1.1 \; m^3$. Take care of significant figures.

        \question Work done by force is $W = \alpha^{2}\beta e^{-\frac{\beta x^{2}}{KT}} $ where $x$ is distance, $K$ is Boltzmann's constant and $T$ is temperature. The dimension of $\alpha$ is
        
        \begin{oneparchoices}
            \choice $ M^{1}L^{2}T^{-1}$
            \choice $ M^{0}L^{1}T^{0}$
            \choice $ M^{0}L^{1}T^{-2}$
            \choice $ M^{2}L^{1}T^{-2}$
        \end{oneparchoices}

        \question \textbf{Statement-1} Resolving power of electron microscope is greater than optical microscope. \\
                    \textbf{Statement-2} de-Broglie wavelength of electron is very less than visible light.

        \begin{parts}
            \setlength{\itemsep}{1pt}
            \setlength{\parskip}{0pt}
            \setlength{\parsep}{0pt}
            \part Statement-1 is correct, Statement-2 is correct but not correct explanation.
            \part Statement-1 is correct, Statement-2 is correct and it is correct explanation. 
            \part Statement-1 is correct, Statement-2 is wrong. 
            \part Statement-1 is wrong, Statement-2 is also wrong. 
        \end{parts}

        \question Two particles of masses $1 \;kg$ and $2 \;kg$ are located at $x = 0$ and $x = 3\;m$. Find the position of their centre of mass.

        \question The position vector of three particles of masses $m_{1}=1\;kg$, $m_{2}=2\;kg$ and $m_{3}=3\;kg$ are
        $r_1 = (\hat{i} + 4\hat{j} + \hat{k})\;m$, $r_2 = (\hat{i} + \hat{j} + \hat{k})\;m$ and
        $r_3 = (2\hat{i} - \hat{j} - 2\hat{k})\;m$ respectively. Find the position vector of their centre of mass.

        \question Four particles of masses $1 \;kg$, $2 \;kg$, $3 \;kg$ and 4 kg are placed at
        the four vertices $A$, $B$, $C$ and $D$ of a square of side $1 \;m$. Find the position of
        centre of mass of the particles.

    \end{questions}
    

	\end{multicols*}	

\end{document}